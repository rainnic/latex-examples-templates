%\selectlanguage{italian}
\begin{abstract}

La redazione di questo documento cerca di inquadrare dal punto di vista geotecnico e ambientale i sistemi geotermici a bassa entalpia, offrendo una visione sintetica della materia e dei relativi pregi e difetti e approfondendo al contempo le teorie termogeologiche che ne stanno alla base.

La geotermia rappresenta una delle tante risorse rinnovabili che la natura offre in maniera costante e duratura, durante tutto l'arco dell'anno. Può essere impiegata come impianto di riscaldamento e raffrescamento per regolare il comfort delle nostre case ed ambienti di lavoro. Può abbattere l'uso delle fonti tradizionali di combustibile fossile e ridurre quindi le emissioni di $CO_2$ e polveri sottili, in ottemperanza agli obblighi presi con il protocollo di Kyoto e le norme comunitarie, nonché la recente certificazione energetica che dal I luglio 2009 è diventata obbligatoria per tutte le unità immobiliari, siano esse ad uso abitativo o lavorativo, di nuova costruzione o ristrutturazione.

In Italia i sistemi a bassa entalpia sono una tecnologia di recente applicazione e costituzione, essendo legati in maniera imprescindibile alle pompe di calore ed a una buona progettazione, che per poter assicurare un rendimento costante e duraturo nel tempo, deve essere rispettosa dell'ambiente circostante ed preservarlo il più possibile. Ovvero in questo caso più che in molti altri, l'interesse privato coincide con l'interesse pubblico. 

Si ha quindi la necessità di fornire validi strumenti di analisi e di sostenibilità, per consentire a questo promettente mercato di svilupparsi in modo sano, corretto e rispettoso dell'ambiente. Per questo motivo si è cercato di analizzare la situazione in Paesi dove questa impiantistica è presente già da diversi decenni, come Svizzera, Germania e Stati Uniti; si sono considerati inoltre atti di convegno, bollettini, opere scientifiche che cercano di affrontare e caratterizzare aspetti chiave della tecnologia ed infine si è evidenziata la normativa italiana vigente in materia. Si sono quindi rilevati e confrontati gli aspetti vincolanti e costruttivi, le criticità tecniche ed i problemi aperti dalla gestione nel medio e lungo termine, adattando il tutto alla situazione idrogeologica della Provincia di Rovigo.

La tesi può essere un buon punto di partenza, rivolto non solo all'ente di controllo che è chiamato a rilasciare il permesso per l'utilizzo della risorsa sottosuolo, ma anche all'utente privato che intende predisporre impianti geotermici ed al professionista incaricato della progettazione e della realizzazione delle opere.

Allo stato attuale comunque manca una chiara ed armonica standardizzazione della materia anche a livello europeo ed ogni Paese, regione o provincia cerca a modo suo di favorirne la crescita, lasciando ampie libertà nelle concessioni\index{concessioni}, con il risultato di avere impianti mal progettati e con impatti negativi sull'ambiente o anche e purtroppo di ostacolarlo, con norme troppo vincolanti e ostruzionistiche che ne limitano lo sviluppo ed il perfezionamento.
\\[1cm]
\textbf{Keyword}: geotecnica, energia, geotermia, legge, sonde
\end{abstract} 

\newpage
\selectlanguage{english}
\begin{abstract}

The editing of the present document is focused on low enthalpy geothermal systems studied from a geotechnical and environmental perspective. At the same time the document offers a concise vision of this subject with its merits and demerits analysing the thermogeological  theories on which these systems are based.

During the year nature constantly and lastingly offers many renewable resources; geothermal energy is one of these resources which can be used as heating and cooling system to regulate the comfort in our houses and business places. Geothermal energy can reduce the use of traditional fossil fuels and thus reduce $\mathrm{CO}_2$ emissions and the production of  particulate matter as provided by the Kyoto Protocol undertakings and EU laws. Moreover from July $1^{\mathrm{st}}$, $2009$ it is compulsory for all buildings, either for domestic or business usage, new or renovated, to have an energy certification.

In Italy the use of low enthalpy systems is very new, as they absolutely need heat pumps and a good planning which must respect and preserve the natural surrounding environment as long as possible in order to assure a constant long-lasting performance. In this case, more than in many others, private interest agrees with the public one.

We need good analysis and governance's instruments to allow this promising market to develop correctly and respecting the environment. For this reason I have tried to analyse the situation in countries like Swiss, Germany and the USA where this type of plant engineering has been installed for several decades; I have also studied convention documents, bulletins and scientific works dealing with key aspects of this technology and finally I have highlighted the Italian law in force concerning this subject. Then I have pointed out and compared the binding and structural aspects, the technical difficulties and the opened problems issued from a middle or long-term management. The whole has been adapted to Rovigo's hydrogeologic situation.

The thesis can be a good starting point addressed not only to the authority, which has to allow the use of ground energy resource, but also to the private consumer who wants to install geothermal systems and to the planner responsible for works' planning and realization.

Nowadays there isn't a clear standardization of this subject. Therefore, on one hand every country, region or district encourages the geothermal systems' development by granting many permissions, with the result that there are bad planned systems with negative effects on the environment. On the other hand the local authorities may unfortunately limit any improvement and development with too much restrictive and obstructionist laws.
\\[1cm]
\textbf{Keywords}: geotechnical, energy, geothermal, regulation, borehole
\end{abstract}
\selectlanguage{italian}