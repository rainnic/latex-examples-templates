\documentclass{fancyslides}
%% Presentation was created using:
%% Fancyslides class, by Pawel Lupkowski
%% http://www.staff.amu.edu.pl/~p_lup/?page_id=1057
%%
%\usepackage{fontspec}

%% Beamer settings (do not change)
\usetheme{default} 
\setbeamertemplate{navigation symbols}{} %no navigation symbols
\setbeamercolor{structure}{fg=\yourowntexcol} 
\setbeamercolor{normal text}{fg=\yourowntexcol} 

%%
%% Extra package
%%

%% Aspect ratio
\newcommand{\aspectW}{16}
\newcommand{\aspectH}{9}
\usepackage[orientation=landscape,size=custom,width=\aspectW,height=\aspectH,scale=0.5,debug]{beamerposter}
\usepackage{comment}

%%
%% GRID & STYLE
%%
\usepackage{tikz}
\usetikzlibrary{arrows,positioning}
%% MY STYLE AND COMMAND FOR TIKZ
\tikzset{tick/.style={below=3pt}}

%%
%% Function to call the grid
%% Usage:
%% \showTheGrid{1} TRUN ON
%% \showTheGrid{0} TRUN OFF
\newcommand{\showTheGrid}[1]{
    \draw[step=1cm,gray,very thin, opacity=#1] (0,0) grid (\aspectW,\aspectH);
    \foreach \x in {0,...,\aspectW}
		\draw [blue, opacity=#1] (\x,4) -- (\x,4) node[below] {\x};
    \foreach \y in {0,...,\aspectH}
		\draw [red, opacity=#1] (8,\y) -- (8,\y) node[right] {\y};
		}

%% Multilanguage support
%% every time one must be activated and the other deactivated
\excludecomment{en}
\includecomment{it}

%%%%%%%%%%%%%%%%%%%%%%%%%
%%% CUSTOMISATIONS %%%%%%
%%%%%%%%%%%%%%%%%%%%%%%%%

% THE FOLLOWING COLOURS ARE PREDEFINED IN THE CLASS
%bi -- WHITE
%cz -- BLACK
%sz -- GRAY
%nieb -- BLUE
%ziel -- GREEN
%pom -- ORANGE
%% YOU CAN DEFINE YOUR OWN COLOUR TO USE HERE. SEE MAN.PDF


%%%% SLIDE ELEMENTS
\newcommand{\structureopacity}{0.75} %opacity for the structure elements (boxes and dots)
\newcommand{\strcolor}{nieb} %elements colour (predefined nieb; pom; ziel)

%%%% TEXT COLOUR
\newcommand{\yourowntexcol}{bi}

%% COMMAND TO FIX THE LOGO POSITION (x,y) in cm
\usepackage[absolute,overlay]{textpos}
\setlength{\TPHorizModule}{1cm}
\setlength{\TPVertModule}{1cm}
\newcommand{\MyLogo}{%
   \begin{textblock}{14}(7.0,3.0)
      \includegraphics[height=6cm,angle=0]{figures/MyLogo}
   \end{textblock}
   }

%%%%%%%%%%%%%%%%%%%%%%%%%
%%% TITLE SLIDE DATA %%%%
%%%%%%%%%%%%%%%%%%%%%%%%%
\newcommand{\titlephrase}{How to crop \& split a PDF \\(in \LaTeX{}, without using it)}
\newcommand{\name}{Nicola Rainiero}
\newcommand{\affil}{\href{http://rainnic.altervista.org}{rainnic.altervista.org}}
\newcommand{\email}{\href{mailto:rainnic@altervista.org}{rainnic@altervista.org}}
\newcommand{\firm}{Rainnic in the clouds}
\title{\titlephrase}
\newcommand{\keyWords}{LaTeX, online, Overleaf, split, crop}

%%
%% URL ADVANCED
%%
\usepackage{hyperref}
\hypersetup{
	pdftitle={\@title},
	pdfsubject={\firm},
	pdfauthor={\name},
	pdfkeywords={\keyWords},
	%pdfpagemode=FullScreen, % once opened it goes in fullscreen mode
	%citecolor=black,
	%filecolor=black,
	%linkcolor=black,
	%urlcolor=black
}

\begin{document}
%\fontspec[Ligatures={TeX}]{Lato} %% SETS THE FONT FOR THE PRESENTATION

%
% Title page
%
\startingslide %this generates titlepage from the data above

%
% Section 1: Introduction
%
\fbckg{figures/6}
\begin{frame}
\MyLogo
\framedsl{A brief introduction}
\end{frame}

\begin{frame}
\MyLogo
\misc{
	\normalsize{I show you how it is simple and powerful the use of \LaTeX{} to split and crop iteratively a PDF in half, just editing a few lines of code, \emph{basically modifying only some numbers}. You can use my \LaTeX{} template in Overleaf or in your editor.}
	\newline \newline
	Here is the link: \textcolor{yellow!100}{\href{http://bit.ly/1F6r5H4}{bit.ly/1F6r5H4}}
	}
\end{frame}

\begin{frame}
\MyLogo
\misc{
	\normalsize{
	\textcolor{white!100}{Online with \emph{Overleaf}: remember that I post a read-only view template. If you want to edit it, you have to paste its content in a new project and register in Overleaf (the only way that allows you to upload PDF files).}
\newline \newline
	Here is my sign up link: \textcolor{yellow!100}{\href{https://www.overleaf.com/signup?ref=03181694e0d0}{www.overleaf.com/signup?ref=03181694e0d0}}
	}}
\end{frame}

\begin{frame}
\MyLogo
\misc{
	\normalsize{
	\textcolor{white!100}{In \emph{your local editor}: Copy and paste its content in your \LaTeX{} editor. I suggest you one with an integrated viewer and live update.}
\newline \newline
Which one? Check the \emph{Comparison of TeX editors} page on Wikipedia: \newline
	\textcolor{yellow!100}{\href{http://bit.ly/1lzWyUu}{en.wikipedia.org/wiki/Comparison\_of\_TeX\_editors}}
	}}
\end{frame}

\fbckg{figures/1}
\begin{frame}
\MyLogo
 \makebox[\textwidth][c]{\begin{tikzpicture}
    \showTheGrid{0}
        \draw [blue, fill=yellow, fill opacity=0.25] (0.55,0.5) rectangle (8.5,8.3) node[red, text opacity=0.95, draw=gray, ultra thin, rounded corners=.25ex, fill=gray!20,text width=5.5cm, text justified,  inner sep=.5ex, below right=1.65,rotate=0] {How it works?\\ \\ It is simple, you have only to edit the area on the left, changing opportunely some numbers.\\ \\In the next sections, I will show you how, with the help of handy examples.};
  \end{tikzpicture}
  }
\end{frame}

%
% Section 2: Let's start
%
\begin{it}
\begin{frame}
\MyLogo
\framedsl{Let's start!}
\end{frame}
\end{it}

\begin{frame}
\MyLogo
 \makebox[\textwidth][c]{\begin{tikzpicture}
    \showTheGrid{0}
    \draw [blue, fill=yellow, fill opacity=0.25] (10,7) rectangle (15,8);
   \draw[tick, blue, -triangle 90] (10,7.5) to node[rotate=-10] {CLICK HERE} (2.75,8.75);
  \end{tikzpicture}
 }
\end{frame}

\fbckg{figures/2}
\begin{frame}
\MyLogo
\end{frame}

\fbckg{figures/1}
\begin{frame}
\MyLogo
 \makebox[\textwidth][c]{\begin{tikzpicture}
    \showTheGrid{0}
    \draw [blue, fill=yellow, fill opacity=0.25] (10,7) rectangle (15,8);
   \draw[tick, blue, -triangle 90] (10,7.5) to node[above, rotate=18] {PDF FILENAME} (3.75,5.5);
    \draw[tick, blue, -triangle 90] (10,7.25) to node[rotate=18] {CORRECT THE SIZE} (3.75,5.25);
     \draw[tick, blue, -triangle 90,rounded corners=5pt] (12.5,7) |- node[above left=0.5, rotate=0] {FIX THE NUMBER OF PAGES} (7.5,1);
  \end{tikzpicture}
 }
\end{frame}

\fbckg{figures/3}
\begin{frame}
\MyLogo
\misc{If your input is correct, you will see in the preview: your PDF and a virtual blue rectangle that cover the first half.}
\end{frame}

%
% Section 2: Example 1: just split it in half
%
\fbckg{figures/4}
\begin{frame}
\MyLogo
\framedsl{Example 1:}
\framedsl{split it in two half}
\end{frame}

\begin{frame}
\MyLogo
 \makebox[\textwidth][c]{\begin{tikzpicture}
    \showTheGrid{0}
    \draw [blue, fill=yellow, fill opacity=0.25] (0.5,7) rectangle (2.65,7.2) node[red, text opacity=0.95, above right, pos=.5] {{\small FIRST STEP, check in the output the result}};
    \draw [blue, fill=yellow, fill opacity=0.25] (11,7) rectangle (14,8) node[red, text opacity=0.95, pos=.5] {CROP FALSE};
        \draw [blue, fill=yellow, fill opacity=0.25] (11,5) rectangle (14,6) node[red, text opacity=0.95, pos=.5] {DOUBLE TRUE};
   \draw[tick, blue, -triangle 90] (11,7.5) to node[above, rotate=18] {} (2.25,4.5);
    \draw[tick, blue, -triangle 90] (11,5.5) to node[rotate=18] {} (2.5,3.3);
  \end{tikzpicture}
  }
\end{frame}

\fbckg{figures/5}
\begin{frame}
\MyLogo
 \makebox[\textwidth][c]{\begin{tikzpicture}
    \showTheGrid{0}
    \draw [blue, fill=yellow, fill opacity=0.25] (0.5,6.85) rectangle (2.65,7.05) node[red, text opacity=0.95, above right, pos=.5] {{\small SECOND STEP, scroll in the preview the result and save it in a new PDF}};
    \draw[tick, red, -triangle 90] (12.75,7.5) to node[rotate=18] {} (6.5,8.75);
    \draw[very thick, blue, open triangle 60-open triangle 60] (15.75,3) to node[rotate=18] {} (15.75,6);   
  \end{tikzpicture}
  }
\end{frame}

%
% Section 3: Example 2: crop and split it in half
%
\fbckg{figures/6}
\begin{frame}
\MyLogo
\framedsl{Example 2:}
\framedsl{crop and split it}
\end{frame}

\begin{frame}
\MyLogo
 \makebox[\textwidth][c]{\begin{tikzpicture}
    \showTheGrid{0}
    \draw [blue, fill=yellow, fill opacity=0.25] (0.5,7) rectangle (2.65,7.2) node[red, text opacity=0.95, above right, pos=.5] {{\small FIRST STEP, check in the output the result}};
    \draw [blue, fill=yellow, fill opacity=0.25] (11,7) rectangle (14,8) node[red, text opacity=0.95, pos=.5] {CROP TRUE};
        \draw [blue, fill=yellow, fill opacity=0.25] (11,5) rectangle (14,6) node[red, text opacity=0.95, pos=.5] {DOUBLE TRUE};
   \draw[tick, blue, -triangle 90] (11,7.5) to node[above, rotate=18] {} (2.25,4.5);
    \draw[tick, blue, -triangle 90] (11,5.5) to node[rotate=18] {} (2.5,3.3);
  \end{tikzpicture}
  }
\end{frame}

\begin{frame}
\MyLogo
 \makebox[\textwidth][c]{\begin{tikzpicture}
    \showTheGrid{0}
    \draw [blue, fill=yellow, fill opacity=0.25] (0.5,7) rectangle (2.65,7.2) node[red, text opacity=0.95, above right, pos=.5] {{\small FIRST STEP, check in the output the result}};
    \draw [blue, fill=yellow, fill opacity=0.25] (2.25,3.95) rectangle (2.75,4.35) node[red, text opacity=0.95, above=0.05] {{\small size of the cutter}};
    \draw[very thick, orange, open triangle 60-open triangle 60] (9.2,6.7) to node[rotate=90, above] {280} (9.2,1.9);
    \draw[very thick, orange, open triangle 60-open triangle 60] (9.4,1.7) to node[rotate=0, below] {150} (11.9,1.7);
        \draw[very thick, orange, open triangle 60-open triangle 60] (12.8,6.7) to node[rotate=90, above] {280} (12.8,1.9);
    \draw[very thick, orange, open triangle 60-open triangle 60] (12.95,1.7) to node[rotate=0, below] {150} (15.5,1.7);
  \end{tikzpicture}
  }
\end{frame}

\begin{frame}
\MyLogo
 \makebox[\textwidth][c]{\begin{tikzpicture}
    \showTheGrid{0}
    \draw [blue, fill=yellow, fill opacity=0.25] (0.5,7) rectangle (2.65,7.2) node[red, text opacity=0.95, above right, pos=.5] {{\small FIRST STEP, check in the output the result}};
    \draw [blue, fill=yellow, fill opacity=0.25] (1.95,1.8) rectangle (2.65,2.65) node[red, text opacity=0.95, align=left, above right] {distance of the two cutters \\ from the bottom left of the page};
    \draw[very thick, orange, <->] (8.9,3) node[orange, above right] {\huge y} -- (8.9,1.9) -- (10,1.9) node[orange, above right] {\huge x};
  \end{tikzpicture}
  }
\end{frame}

\begin{frame}
\MyLogo
 \makebox[\textwidth][c]{\begin{tikzpicture}
    \showTheGrid{0}
    \draw [blue, fill=yellow, fill opacity=0.25] (0.5,7) rectangle (2.65,7.2) node[red, text opacity=0.95, above right, pos=.5] {{\small FIRST STEP, check in the output the result}};
    \draw [blue, fill=yellow, fill opacity=0.25] (1.95,1.8) rectangle (2.65,2.65) node[red, text opacity=0.95, align=left, above right] {distance of the two cutters \\ from the bottom left of the page};
    \draw[thick, orange, open triangle 60-open triangle 60] (8.9,2.5) to node[rotate=0, above] {30} (9.4,2.5);
    \draw[thick, orange, open triangle 60-open triangle 60] (8.8,1.92) to node[rotate=0, below] {242} (12.95,1.92);
  \end{tikzpicture}
  }
\end{frame}

\fbckg{figures/7}
\begin{frame}
\MyLogo
 \makebox[\textwidth][c]{\begin{tikzpicture}
    \showTheGrid{0}
    \draw [blue, fill=yellow, fill opacity=0.25] (0.5,6.85) rectangle (2.65,7.05) node[red, text opacity=0.95, below right] {{\small SECOND STEP, scroll in the preview the result and save it in a new PDF}};
    \draw[thick, red, -triangle 90] (13.7,7) to node[rotate=18] {} (6.5,8.75);
    \draw[very thick, blue, open triangle 60-open triangle 60] (15.75,3) to node[rotate=18] {} (15.75,6);   
  \end{tikzpicture}
  }
\end{frame}

%
% Section 4: Example 3: crop and split it in half
%
\fbckg{figures/8}
\begin{frame}
\MyLogo
\framedsl{Example 3:}
\framedsl{only crop it}
\end{frame}

\begin{frame}
\MyLogo
 \makebox[\textwidth][c]{\begin{tikzpicture}
    \showTheGrid{0}
    \draw [blue, fill=yellow, fill opacity=0.25] (0.5,7) rectangle (2.65,7.2) node[red, text opacity=0.95, above right, pos=.5] {{\small FIRST STEP, check in the output the result}};
    \draw [blue, fill=yellow, fill opacity=0.25] (11,7) rectangle (14,8) node[red, text opacity=0.95, pos=.5] {CROP TRUE};
        \draw [blue, fill=yellow, fill opacity=0.25] (11,5) rectangle (14,6) node[red, text opacity=0.95, pos=.5] {DOUBLE FALSE};
   \draw[tick, blue, -triangle 90] (11,7.5) to node[above, rotate=18] {} (2.25,4.5);
    \draw[tick, blue, -triangle 90] (11,5.5) to node[rotate=18] {} (2.5,3.3);
  \end{tikzpicture}
  }
\end{frame}

\fbckg{figures/9}
\begin{frame}
\MyLogo
 \makebox[\textwidth][c]{\begin{tikzpicture}
    \showTheGrid{0}
    \draw [blue, fill=yellow, fill opacity=0.25] (0.5,7) rectangle (2.65,7.2) node[red, text opacity=0.95, above right, pos=.5] {{\small FIRST STEP, check in the output the result}};
    \draw [blue, fill=yellow, fill opacity=0.25] (2.25,3.95) rectangle (2.75,4.35) node[red, text opacity=0.95, above=0.05] {{\small size of the cutter}};
    \draw[very thick, orange, open triangle 60-open triangle 60] (9.2,6.35) to node[rotate=90, above] {280} (9.2,1.65);
    \draw[very thick, orange, open triangle 60-open triangle 60] (9.4,1.5) to node[rotate=0, below] {360} (15.45,1.5);
  \end{tikzpicture}
  }
\end{frame}

\begin{frame}
\MyLogo
 \makebox[\textwidth][c]{\begin{tikzpicture}
    \showTheGrid{0}
    \draw [blue, fill=yellow, fill opacity=0.25] (0.5,7) rectangle (2.65,7.2) node[red, text opacity=0.95, above right, pos=.5] {{\small FIRST STEP, check in the output the result}};
    \draw [blue, fill=yellow, fill opacity=0.25] (1.95,2.2) rectangle (2.65,2.65) node[red, text opacity=0.95, align=left, above right] {distance of the first cutter \\ from the bottom left of the page};
    \draw[very thick, orange, <->] (8.9,2.6) node[orange, above right] {\huge y} -- (8.9,1.6) -- (9.9,1.6) node[orange, above right] {\huge x};
  \end{tikzpicture}
  }
\end{frame}

\begin{frame}
\MyLogo
 \makebox[\textwidth][c]{\begin{tikzpicture}
    \showTheGrid{0}
    \draw [blue, fill=yellow, fill opacity=0.25] (0.5,7) rectangle (2.65,7.2) node[red, text opacity=0.95, above right, pos=.5] {{\small FIRST STEP, check in the output the result}};
    \draw [blue, fill=yellow, fill opacity=0.25] (1.95,2.2) rectangle (2.65,2.65) node[red, text opacity=0.95, align=left, above right] {distance of the first cutter \\ from the bottom left of the page};
    \draw[thick, orange, open triangle 60-open triangle 60] (8.9,1.6) to node[rotate=0, above] {30} (9.4,1.6);
    \draw[thick, orange, <->] (9.8,1.5) to node[rotate=0, right] {3} (9.8,1.7);
  \end{tikzpicture}
  }
\end{frame}

\fbckg{figures/10}
\begin{frame}
\MyLogo
 \makebox[\textwidth][c]{\begin{tikzpicture}
    \showTheGrid{0}
    \draw [blue, fill=yellow, fill opacity=0.25] (0.5,6.85) rectangle (2.65,7.05) node[red, text opacity=0.95, above right=0.5] {{\small SECOND STEP, scroll in the preview the result and save it in a new PDF}};
    \draw[thick, red, -triangle 90] (14.1,8) to node[rotate=18] {} (6.5,8.75);
    \draw[very thick, blue, open triangle 60-open triangle 60] (15.75,3) to node[rotate=18] {} (15.75,6);   
  \end{tikzpicture}
  }
\end{frame}

%
% Thank you
%
\fbckg{figures/6}
\begin{frame}
\MyLogo
  \thankyou   %%%% ending slide with thank you notice
\end{frame}

%
% Sources
%
\fbckg{figures/blank}
\begin{frame}
%\MyLogo
\sources{
\includegraphics[width=0.10\textwidth]{figures/github-collab-retina-preview} \ \href{http://bit.ly/1bc7KsA}{bit.ly/1bc7KsA}\\
\includegraphics[width=0.10\textwidth]{figures/logo} \ \href{http://rainnic.altervista.org/tag/latex}{rainnic.altervista.org/tag/latex}

\vspace{1.0cm}

\noindent

Presentation was created using: \textit{\href{http://bit.ly/1Ej1whM}{Fancyslides class} by Pawel Lupkowski}
}
\end{frame}

\end{document}