%\title{How to split a PDF in half}
%% Original version by J.S.Milne 25.09.05
%% (two2one.tex - http://www.jmilne.org/math/Books/scan.html)
%%
%%
%% 					Updated by Nicola Rainiero
%% 					http://rainnic.altervista.org/it/tag/latex
%%					------------------------------------------
%%
%% Read me:
%% - Load your PDF (Project --> files)
%% - Change opportunely the size and the number of pages
%% - To discovering the correct size, you can use software like:
%%   Inkscape, Gimp or a PDF Reader
%% - I put the units in mm but you can choose in whatever you want
%% - Activate a step at a time
%%

\documentclass[geometry]{minimal}
\usepackage{mathptmx}
\usepackage[12pt]{moresize}
\usepackage{comment}
\usepackage{ifthen}
\usepackage{pdfpages}
\usepackage{tikz}


% <---------------------------- EDIT THIS AREA ONLY ----------------------------------->

% Default style and colour of the cutter
\tikzset{marker/.style={draw=blue,fill=blue, opacity=0.25}}

%% Every time one of the following must be activated and the others deactivated
%% \includecomment{...} --> means turn ON
%% \excludecomment{...} --> means turn OFF
\includecomment{first} 		% for discovering the dimension of the cuts
\excludecomment{second}		% for printing a cleaned and trimmed PDF
\excludecomment{third}		% for joining the pages of the II stage without white spaces
							% (save and use the new PDF before to do it)

%%
%% Define the file name and the dimension of your PDF
%% Put the size in mm
%% The typical size of an A4 document is 210x297 mm
\newcommand\units{mm}					% <--- set the units (default mm)
\newcommand\fileName{two_side.pdf}		% <--- the name of the PDF to process
\newcommand\width{420} 					% <--- the width size in mm of your PDF
\newcommand\height{297} 				% <--- the height size in mm of your PDF

%%
%% 							Do you want to crop the white spaces around the PDF?
%%
\newcommand\crop{1} 					% <-- "0" FALSE "1" TRUE
% If true put the dimension of the virtual rectangle that covers the pages
\newcommand\cutterW{150} 				% <--- the width size in mm of your cutter
\newcommand\cutterH{280} 				% <--- the height size in mm of your cutter

%%
%% 							Do you want to cut two sections per page?
%%
\newcommand\double{1} 					% <--- "0" FALSE "1" TRUE
%% If true put x_i, y_i the coordinates for moving the virtual rectangle in the page
%% the system start at the bottom left of the whole page
%% YOU CAN PUT ONLY INTEGER NUMBER
%% first rectangle in page (generally odd) or the only one
\newcommand\xOdd{30}					% <-- x_1
\newcommand\yOdd{3}						% <-- y_1
%% second rectangle in page (generally even)
\newcommand\xEven{242}					% <-- x_2
\newcommand\yEven{3}					% <-- x_2

%%
%% 							How many pages do you want to process?
%% default pages + 1, pages +0 if there are two page increment in the 2 step
\newcommand\numberPages{6} 				% <--- the numbers of pages
\newcommand\firstPage{1} 				% <--- fix the first page to start

% <---------------------------- EDIT THIS AREA ONLY  ----------------------------------->

\newcommand\DxOdd{\the\numexpr\xOdd+\cutterW}
\newcommand\DyOdd{\the\numexpr\yOdd+\cutterH}
\newcommand\DxEven{\the\numexpr\xEven+\cutterW}
\newcommand\DyEven{\the\numexpr\yEven+\cutterH}    
\newcommand\halfW{\the\numexpr\width/2}
\newcommand\halfH{\the\numexpr\height/2}

\begin{first}
   %% The dimension of the original PDF
   \usepackage[paperwidth=\width \units,paperheight=\height \units]{geometry}
\end{first}

\begin{second}
   \ifthenelse{\crop=0}
      {\usepackage[paperwidth=\halfW mm,paperheight=\height mm]{geometry}}
      {\usepackage[paperwidth=\cutterW \units,paperheight=\cutterH \units]{geometry}}
\end{second}

\begin{third}
   %% Put opportunely the dimension of the final page
   %% it must be an exact multiple of the box discovered in the first step
   %% for example if you want to join perfectly, losing the white spaces
   \usepackage[paperwidth=\the\numexpr\cutterW*2 \units,paperheight=\cutterH \units]{geometry}
\end{third}

\newcounter{pg} %counter of pages
\begin{document}

%% Check if the PDF exists
\IfFileExists{\fileName}
   {\includecomment{checkTrue}}
   {\excludecomment{checkTrue} 
      \begin{center}
         {\HUGE LOAD A PDF FIRST!}
         \par \addvspace \bigskipamount \par
         {\HUGE Put the correct name and size in the edit area.}
         \par \addvspace \bigskipamount \par
         {\HUGE Correct the other options and enjoy!}
      \end{center}
   }

\begin{checkTrue}
%\InputIfFileExists{two_side.pdf}{
% -------------------------
% 				FIRST STEP:
% -------------------------
% Size and draw the boxes in the original PDF
% To activated it: \includecomment{first}
\ifthenelse{\crop=0}
   {\newcommand{\mycbox}{\tikz{\path[marker] (0,0) rectangle (\halfW \units,\height \units);}}}
   {\newcommand{\mycbox}{\tikz{\path[marker] (0,0) rectangle (\cutterW \units, \cutterH \units);}}}

\begin{first}
   %% pages={-} means all
   %% pages={1-5} means the interval between 1 to 5
   \includepdf[pages={-}, picturecommand={
      \setlength{\unitlength}{1.0 \units}
      \ifthenelse{\crop=0}
         {\put(0,0){\mycbox}
          \ifthenelse{\double=1}{\put(\halfW,0){\mycbox}}{}}
         {\put(\xOdd,\yOdd){\mycbox}
          \ifthenelse{\double=1}{\put(\xEven,\yEven){\mycbox}}{}}
   }]{\fileName}
\end{first}


% --------------------------
% 				SECOND STEP:
% --------------------------
% Draw the pages using the coordinates and the size discovered in the first step
% To activated it: \includecomment{second}
%
% Save the PDF output
%
\begin{second}
   \setcounter{pg}{\firstPage} %fix the first page to start
   \whiledo{\value{pg}<\numberPages}{% default pages+1, pages+0 for two increment
      \ifthenelse{\crop=1}
       {\includepdf[pages=\thepg,viewport=\xOdd mm \yOdd mm \DxOdd mm \DyOdd mm]{\fileName}
        \ifthenelse{\double=1}{\includepdf[pages=\thepg,viewport=\xEven mm \yEven mm \DxEven mm \DyEven mm]{\fileName}}{}}
       {\ifthenelse{\double=1}{\includepdf[pages=\thepg,viewport=0mm 0mm \halfW mm \height mm]{\fileName}
        \includepdf[pages=\thepg,viewport=\xEven mm \yEven mm \DxEven mm \DyEven mm]{\fileName}}
       {\includepdf[pages=\thepg,viewport=0mm 0mm \halfW mm \height mm]{\fileName}}}      
      %page increment
      \addtocounter{pg}{1}
   }
\end{second}

% -------------------------
% 				THIRD STEP:
% -------------------------
% Useful if you want to join a cropped PDF
%
% Usage: nup=2x1 means print 2 columns x 1 row (= 2 pages per page)
% To activated it: \includecomment{third}
%
% Save the PDF output of the II stage and load it (Project --> files)
\begin{third}
   \includepdf[pages=-, nup=2x1]{split-pdf-half.pdf}%<-your last PDF
\end{third}
\end{checkTrue}

\end{document}